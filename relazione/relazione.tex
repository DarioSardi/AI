\documentclass[12pt]{article}
\usepackage[english]{babel}
\usepackage[utf8x]{inputenc}
\usepackage{amsmath}
\usepackage{graphicx}
\usepackage[colorinlistoftodos]{todonotes}
\usepackage{listings}
\usepackage{caption}
\usepackage{parskip} %dannati indent
\lstset{
    numbers=left,
    breaklines=true,
    tabsize=2,
    basicstyle=\ttfamily,
}

\begin{document}

\begin{titlepage}

\newcommand{\HRule}{\rule{\linewidth}{0.5mm}} % Defines a new command for the horizontal lines, change thickness here

\center % Center everything on the page
 
%----------------------------------------------------------------------------------------
%	HEADING SECTIONS
%----------------------------------------------------------------------------------------

\textsc{\LARGE Università di Bergamo}\\[1.5cm] % Name of your university/college
\textsc{\Large Relazione progetto}\\[0.5cm] % Major heading such as course name
\textsc{\large Artificial intelligence}\\[0.5cm] % Minor heading such as course title

%----------------------------------------------------------------------------------------
%	TITLE SECTION
%----------------------------------------------------------------------------------------

\HRule \\[0.4cm]
{ \huge \bfseries Reti neurali in python}\\[0.3cm] % Title of your document
\HRule \\[1.5cm]
 
\begin{minipage}{0.4\textwidth}
\begin{flushleft} \large
\emph{Autore:}\\
Dario \textsc{Sardi} % Your name
\end{flushleft}
\end{minipage}
~
\begin{minipage}{0.4\textwidth}
\begin{flushright} \large
\emph{Supervisore:} \\
Francesco \textsc{Trovò} % Supervisor's Name
\end{flushright}
\end{minipage}\\[2cm]
{\large 22 Aprile 2019}\\[2cm] % Date, change the \today to a set date if you want to be precise
\includegraphics[scale=0.5]{logo.png}\\[1cm] % Include a department/university logo - this will require the graphicx package
\vfill % Fill the rest of the page with whitespace

\end{titlepage}
%#######################################################################
\begin{abstract}
L'obiettivo del progetto è quello di creare da zero una rete neurale in python senza sfruttare librerie gia esistenti.
\\Si è creato dapprima un percettrone e successivamente una rete neurale con un solo hidden layer.
\end{abstract}
\section{Percettrone}
Per iniziare e prender pratica con eventuali librerie matematiche è stato creato un percettrone, un neurone in grado di compiere semplici scelte binarie.
In quanto classificatore lineare il dataset per il percettrone consiste in una nuvola di punti posizionati randomicamente e pre-classificati in due categorie in base a una funzione lineare stabilita.
\begin{lstlisting}
def function(x):
	m=-1/3
	c=0.5
	return m*x+c

def genFunction(x,y):
	if y>function(x): return 1
	else: return -1



class point:
    def __init__(self,x,y,b):
		self.pos=[x,y,b]
		self.group=genFunction(x,y)

\end{lstlisting}
In questo modo inizializzando un punto con posizione randomica, la sua appartenenza alle classi \{1,-1\} viene determinata dalla sua posizione relativa alla funzione lineare.

\end{document}